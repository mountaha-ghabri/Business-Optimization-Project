
While the phosphate industry in Gabès is a pillar of Tunisia’s industrial exports and a 3\% contributor to the national GDP, its legacy is defined by a severe socio-environmental crisis. Decades of intensive operations have resulted in significant marine pollution and hazardous air emissions, affecting approximately 8,500 local residents. As environmental mandates and constitutional health rights evolve, the status quo is no longer tenable, necessitating a strategic relocation of the industrial complex.

Relocating such a facility is a “wicked problem" requiring the reconciliation of conflicting objectives: minimizing human hazard exposure, containing fiscal costs, ensuring water security in an arid climate, and preserving ecological integrity. These decisions are further complicated by parameter uncertainty, where data for capital outlays and hydrogeological capacities are often indicative rather than absolute.

This study develops a robust decision-support framework to optimize site selection from a nationwide pool of candidates. The model utilizes Multi-Objective Goal Programming (MOGP) integrated with stochastic Monte Carlo simulation to minimize deviations from four primary goals: Population Safety, Economic Feasibility, Water Security, and Environmental Integrity. By adopting a lexicographic hierarchy, the study explicitly quantifies the trade-offs between public safety and expenditure, providing a transparent path toward sustainable industrial development.

\vspace{0.2cm}

The primary aim of this paper is to provide a mathematically rigorous justification for relocation. To achieve this, the project addresses \textbf{three central research questions:}
\begin{enumerate}
    \item Which candidate site emerges as the robust optimal choice under a safety-first, non-compensatory policy?
    \item How resilient is the optimal solution to shifts in fiscal priorities or resource variations?
    \item What is the quantified “price of safety" when balancing population risk reduction against capital investment?
\end{enumerate}

By placing the right to health at the apex of a lexicographic hierarchy, this research provides a transparent mechanism for policy-makers to navigate the transition from industrial pollution toward sustainable development.

\vspace{0.2cm}

The remainder of this report is organized as follows. \textbf{Section \ref{sec:lit} }reviews the theoretical foundations of facility location and multi-objective optimization. \textbf{Section \ref{sec:method} }details the MOGP methodology and the Monte Carlo simulation workflow. \textbf{Section \ref{sec:data} } presents the site-specific data and parameter estimation for the feasible candidates. \textbf{Section \ref{sec:res}} delivers the optimization results and comparative visualizations. Finally, \textbf{Section \ref{sec:dis}} discusses managerial implications and limitations, followed by concluding recommendations in \textbf{Section \ref{sec:conc}}.