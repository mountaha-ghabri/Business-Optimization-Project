\label{sec:conc}

This research formulated a decision-support framework for the strategic relocation of the Gabès phosphate facility, addressing conflicting priorities and data uncertainty. By integrating stochastic screening of 24 candidates with a lexicographic Multi-Objective Goal Programming (MOGP) model, the study demonstrates a transition from cost-centric industrial planning to a model centered on human safety and environmental resilience.

\section*{Summary of Insights and Recommendations}
The analysis yields three critical insights for industrial governance:
\begin{itemize}
    \item \textbf{The Primacy of Safety:} \textbf{Boughrara (Medenine)} is the optimal site, enabling a 97\% reduction in population hazard exposure. The 506~M TND “safety premium" over the status quo is mathematically justified by the lexicographic priority of public health.
    \item \textbf{Obsolescence of the Status Quo:} The current Gabès location is a “dominated" solution in all scenarios. It fails to meet modern environmental and safety thresholds, making continued operations indefensible regardless of fiscal savings.
    \item \textbf{Stability Under Uncertainty:} Monte Carlo simulations confirm that Boughrara remains the optimal choice even under pessimistic cost and resource percentiles, establishing it as a robust long-term strategic recommendation.
\end{itemize}



\section*{Future Applications and Scalability}
The methodology provides a scalable template for “NIMBY" infrastructure challenges, offering a rigorous tool for decision-makers in several areas:
\begin{itemize}
    \item \textbf{Infrastructure Siting:} Applicable to waste-to-energy plants, chemical hubs, and desalination facilities requiring the reconciliation of technical feasibility with community health.
    \item \textbf{Policy Calibration:} The model allows agencies to "stress-test" policy weights, visually demonstrating the consequences of prioritizing fiscal austerity over environmental quality.
    \item \textbf{Data-Poor Environments:} The integration of Monte Carlo methods provides a blueprint for optimization in regions lacking precise public data, offering a statistically defensible path forward despite information gaps.
\end{itemize}

In conclusion, the framework proves that the relocation of the Gabès complex \textit{is the only viable path} to aligning Tunisia's industrial growth with its constitutional mandates for health and environmental protection.