\label{sec:dis}
\section{Managerial and Practical Implications}
The central managerial implication is the clear dominance of \textbf{Boughrara (Medenine)} when public safety is treated as a non-negotiable objective. The model demonstrates that a \textbf{97\% reduction in population exposure} is achievable, but only by accepting a substantial \textbf{506~M TND safety premium} relative to the Gabès status quo. This cost is not a modeling artifact but the quantified price of aligning industrial siting decisions with modern health and environmental standards.

Crucially, the framework makes trade-offs explicit. Lower-cost alternatives, such as Bouaguereb or Batten Ghazal, reduce expenditure but retain materially higher residual exposure, in some cases an order of magnitude larger than Boughrara. Managers can therefore use the model to justify higher capital commitments by demonstrating that cost minimization directly translates into elevated public risk, rather than abstract efficiency gains.

\section{Robustness and Validity}
The recommendation is supported by a rigorous, multi-stage screening process that reduced 24 initial candidates to 6 feasible sites. Across sensitivity tests, \textbf{Boughrara remains optimal in 5 of 6 weight configurations}, losing dominance only under extreme fiscal prioritization. Monte Carlo analysis further confirms feasibility under pessimistic cost and water availability realizations, indicating that the solution is structurally stable rather than parameter-driven.

\section{Limitations and Potential Improvements}
The model is inherently \textbf{static}, evaluating a single relocation decision rather than a phased transition that could smooth fiscal impacts. In addition, all constraints are treated as \textbf{hard thresholds}, whereas real-world negotiations may allow temporary or conditional relaxations. Finally, reliance on adjusted 2014 census data introduces demographic uncertainty that could be reduced using higher-resolution spatial population datasets.

\section{Extensions and Future Research}
Future work should extend the framework to \textbf{multi-period optimization} to capture transition dynamics and overlapping operations. Integrating \textbf{GIS-based site generation} would allow identification of optimal greenfield locations beyond administrative boundaries. Expanding the environmental dimension to include \textbf{life-cycle assessment (LCA)} would also enable evaluation of carbon trade-offs associated with relocation-induced transport changes.
