This study addresses the demand for urgent relocation of the Gabès phosphate processing facility in Tunisia, a move urged due to severe environmental degradation and escalating public health risks. The decision-making process involves evaluating an initial pool of 24 candidate sites, lowest-density municipality in each governorate, filtered through stochastic screening to a final set of six feasible locations. Each site presents distinct trade-offs between population safety, economic costs, water resource availability, and environmental quality.

We develop a multi-objective goal programming model utilizing lexicographic priorities. The model assigns the highest weight to the minimization of population exposure, followed by water availability, cost efficiency, and environmental quality. To ensure operational viability, the model incorporates hard constraints on budget (750 million TND), minimum water requirements (2.0 Mm³/y), and environmental quality thresholds (6.0/10). Monte Carlo simulation with 1,000 iterations per site is employed to address parameter uncertainty, utilizing probability distributions derived from engineering benchmarks and regional data.



The analysis identifies Boughrara (Medenine) as the optimal site. It achieves a 97\% reduction in population exposure while maintaining full compliance with all technical and environmental constraints. Although the optimal solution requires a 506 million TND cost increase over the status quo, the lexicographic framework demonstrates that the substantial public health benefits significantly outweigh the fiscal deviation. Sensitivity analysis confirms the stability of this result, as Boughrara remains the optimal choice across five of six weight scenarios and all tested parameter variations ($\pm$20\%).

The findings indicate that continued operations at the Gabès site are mathematically and ethically unsupportable under modern standards. Relocation to Boughrara represents a robust, policy-aligned solution that prioritizes public safety without compromising operational feasibility. This research provides a transparent methodology for complex industrial relocation problems, although real-world implementation should follow field data validation and expert consultation to supplement the simulated parameters.
